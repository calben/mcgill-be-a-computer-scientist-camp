\documentclass[11pt]{scrartcl}

\usepackage[nochapters]{classicthesis}
\usepackage{graphicx}
\usepackage{abstract}
\usepackage{amsmath}
\usepackage{subfig}
\usepackage{wrapfig}
\usepackage{mathtools}
\usepackage[dvipsnames]{xcolor}
\usepackage{dirtree}
\usepackage{minted}

\renewcommand{\sffamily}{\rmfamily}

\usepackage[top=1.25in, left=0.75in, right=0.75in, bottom=1.25in]{geometry}
\usepackage{multicol}
\setlength\columnsep{2em} % This is the default columnsep for all pages

\usepackage{hyperref}
\hypersetup{
    pdfborder = {0 0 0},
    colorlinks,
    citecolor=Maroon,
    filecolor=Darkgreen,
    linkcolor=Cyan 
}

\title{Day 1 Programming Exercises}
\author{Game Dev Camp}

\begin{document}
	
	\maketitle
	
	\section{Variables and Types}
	
	\subsection{Using the Right Types}
	
		Write a program that stores the following values with the right types:
		
		\begin{enumerate}
			\item 127
			\item 127.0
			\item "All generalizations are dangerous, even this one."
			\item true
		\end{enumerate}
	
	\subsection{This Then That}
	
		You thought you had a favourite number, but now you're not so sure.
		Make a number called "myFavourite."
		Then change it to something else.
		
	\section{Basic Math}
	
		Here's a reminder of basic math in Java:
		
		\inputminted{java}{code-snippets/math.java}
		
		Write a program that makes a number count down from 10 to 0 by reassigning the number.
		
	\section{If Statements}
	
		\inputminted{java}{code-snippets/ifstatements.java}
		
		Your ship is 10 units tall.
		An asteroid is 40 units tall.
		Both a ship and an asteroid have a center, given to you as shipYPosition and asteroidYPosition.
		How would you tell a ship to dodge in the direction that requires the least movement?
		That is to say if the ship is a little lower than the asteroid, it should dodge beneath it.
		If the ship is a little higher than the asteroid, it should dodge above it.
		If it isn't going to hit the asteroid, the ship doesn't need to move.
		
	\section{Printing Things to the Screen}
	
		You can print things to the screen by doing the following:
		
		\inputminted{java}{code-snippets/screenprinting.java}
		
		Make another countdown from 10 to 0, but this time have it print to screen!
	
\end{document}
